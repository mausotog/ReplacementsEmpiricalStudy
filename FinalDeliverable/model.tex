% !TEX root = FinalDeliverable.tex

\section{Building the model}

In order to build a probabilistic model of statement replacements in bug fixes, we analyzed the 20 most recent bug-fixing commits in each of the 100 projects that we have selected for this study. The script for building the model by analyzing the commits took 4 days to run to completion on a laptop with an Intel Core i7 processor and 8GB RAM.

As mentioned in section~\ref{sec:statement-replacements}, we used QACrashFix to determine which statements have been replaced with which other statements. Since QACrashFix uses the Eclipse Java Development Tools (JDT)\footnote{http://www.eclipse.org/jdt/} to parse code snippets, we have used the statement types used in the Abstract Syntax Tree (AST) generated by JDT as the statement types in our model. The JDT classifies statements in the Java language into 22 categories. We based our model and results on these 22 statement types. Since the names of the statement types are long, we have abbreviated them for easy reference. The statement types and their respective abbreviations are as follows:

\begin{verbatim}
As: AssertStatement
Bl: Block
Br: BreakStatement
CI: ConstructorInvocation
Co: ContinueStatement
Do: DoStatement
Em: EmptyStatement
EF: EnhancedForStatement
Ex: ExpressionStatement
Fo: ForStatement
If: IfStatement
La: LabeledStatement
Re: ReturnStatement
SC: SuperConstructorInvocation
Ca: SwitchCase
Sw: SwitchStatement
Sy: SynchronizedStatement
Th: ThrowStatement
Tr: TryStatement
TD: TypeDeclarationStatement
VD: VariableDeclarationStatement
Wh: WhileStatement
\end{verbatim}

We built the probabilistic model by determining the instances in which statements of a particular type (replacee type) have been replaced to fix a bug. Among those instances, we determined the type of the statement being used as the replacement statement (replacer type). We then counted the number of times a statement of the replacer type has been used to replace a statement of the replacee type. By following this process by considering each of the 22 statement types described above as the replacee type, we built the model that is shown in Table~\ref{tab:likelihood}. The statement types along the rows of the table are replacee types and the statement types along the columns are replacer types. The values shown in the table are percentages. For example, the first row of the table indicates that in all cases where an assert statement was replaced to fix a bug, it was replaced by an expression statement 25\% of the time, by an ``if" statement 25\% of the time, by a return statement 25\% of the time and by a variable declaration statement 25\% of the time.

\begin{table*}
	\centering
	\resizebox{\textwidth}{!}{
	%\renewcommand{\arraystretch}{0.9}% Tighter
		\begin{tabular}{l|rrrrrrrrrrrrrrrrrrrrrr} \toprule
 				&\texttt{As}&\texttt{Bl}&\texttt{Br}&\texttt{CI}&\texttt{Co}&\texttt{Do}&\texttt{Em}&\texttt{EF}&\texttt{Ex}&\texttt{Fo}&\texttt{If}&\texttt{La}&\texttt{Re}&\texttt{SC}&\texttt{Ca}&\texttt{Sw}&\texttt{Sy}&\texttt{Th}&\texttt{Tr}&\texttt{TD}&\texttt{VD}&\texttt{Wh}
		

			\\ \midrule
\texttt{As}	&0&	0&	0&	0&	0&	0&	0&	0&	25&	0&	25&	0&	25&	0&	0&	0&	0&	0&	0&	0&	25&	0\\
\texttt{Bl}	&0&	0&	0&	1&	0&	0&	0&	0&	33&	0&	15&	0&	31&	0&	0&	0&	0&	0&	0&	0&	19&	0\\
\texttt{Br}	&0&	0&	0&	0&	0&	0&	0&	0&	48&	0&	12&	0&	0&	0&	40&	0&	0&	0&	0&	0&	0&	0\\
\texttt{CI}	&0&	10&	0&	0&	0&	0&	0&	0&	0&	0&	0&	0&	10&	80&	0&	0&	0&	0&	0&	0&	0&	0\\
\texttt{Co}	&0&	0&	0&	0&	0&	0&	0&	0&	0&	0&	0&	0&	100&	0&	0&	0&	0&	0&	0&	0&	0&	0\\
\texttt{Do}	&0&	0&	0&	0&	0&	0&	0&	0&	100&	0&	0&	0&	0&	0&	0&	0&	0&	0&	0&	0&	0&	0\\
\texttt{Em}	&0&	0&	0&	0&	0&	0&	0&	0&	0&	0&	0&	0&	0&	0&	0&	0&	0&	0&	0&	0&	0&	0\\
\texttt{EF}	&0&	12&	0&	0&	0&	0&	0&	0&	54&	4&	12&	0&	0&	0&	0&	0&	0&	0&	12&	0&	8&	0\\
\texttt{Ex}	&0&	26&	3&	0&	0&	0&	0&	1&	0&	0&	13&	0&	3&	0&	0&	0&	0&	1&	3&	0&	50&	0\\
\texttt{Fo}	&0&	8&	0&	0&	0&	0&	0&	17&	0&	0&	0&	0&	75&	0&	0&	0&	0&	0&	0&	0&	0&	0\\
\texttt{If}	&0&	19&	2&	0&	0&	0&	0&	1&	52&	0&	0&	0&	9&	0&	0&	0&	0&	0&	3&	0&	14&	0\\
\texttt{La}	&0&	0&	0&	0&	0&	0&	0&	0&	0&	0&	0&	0&	0&	0&	0&	0&	0&	0&	0&	0&	0&	0\\
\texttt{Re}	&0&	2&	0&	2&	5&	0&	0&	0&	43&	0&	22&	0&	0&	2&	2&	0&	0&	3&	2&	0&	19&	0\\
\texttt{SC}	&0&	8&	0&	54&	0&	0&	0&	0&	23&	0&	0&	0&	0&	0&	0&	0&	0&	0&	0&	0&	15&	0\\
\texttt{Ca}	&0&	0&	30&	0&	0&	0&	0&	0&	20&	0&	10&	0&	30&	0&	0&	0&	0&	0&	0&	0&	10&	0\\
\texttt{Sw}	&0&	33&	0&	0&	0&	0&	0&	0&	0&	0&	67&	0&	0&	0&	0&	0&	0&	0&	0&	0&	0&	0\\
\texttt{Sy}	&0&	0&	0&	0&	0&	0&	0&	0&	10&	0&	20&	0&	0&	0&	0&	0&	0&	0&	70&	0&	0&	0\\
\texttt{Th}	&0&	0&	0&	0&	0&	0&	0&	0&	71&	0&	0&	0&	29&	0&	0&	0&	0&	0&	0&	0&	0&	0\\
\texttt{Tr}	&0&	0&	0&	0&	0&	0&	0&	0&	26&	0&	9&	0&	4&	0&	0&	0&	30&	0&	0&	0&	30&	0\\
\texttt{TD}	&0&	0&	0&	0&	0&	0&	0&	0&	0&	0&	0&	0&	0&	0&	0&	0&	0&	0&	0&	0&	0&	0\\
\texttt{VD}	&0&	6&	0&	0&	0&	0&	0&	1&	81&	1&	6&	0&	2&	0&	0&	0&	0&	0&	2&	0&	0&	0\\
\texttt{Wh}	&0&	0&	0&	0&	0&	0&	0&	0&	0&	0&	0&	0&	0&	0&	0&	0&	0&	0&	0&	0&	0&	0\\
			\\ \bottomrule
		\end{tabular}
		}
\caption{Likelihood of replacing a statement type (row) by a statement of another type (column), for Java. [Numbers in the table are percentages.]}
\label{tab:likelihood}
\end{table*}
