% !TEX root = FinalDeliverable.tex

\section{Introduction}

A considerable amount of research has been done in the area of automatic bug repair~\cite{kim2013, legoues2012, Mechtaev15, Long2016, weimer2009, kai} with the goal of developing automated techniques to repair errors in software without requiring human intervention. One of the more successful approaches so far has been GenProg~\cite{weimer2009, legoues2012}, a tool which combines stochastic search methods like genetic programming with lightweight program analyses to find patches for real bugs in software. Based on evolutionary computation, GenProg applies three different kinds of possible changes to code in order to find a patch for a given bug. These changes are: (1) delete, (2) replace, and (3) append. However, GenProg is limited by the fact that it is possible to generate candidate patches which fail to compile because it applies the mutation operators (delete, replace and append) in a random manner.

In addition, a considerable proportion of these works target programs written in C. Indeed, researchers targeting Java often start from or compare against Java-based implementations of techniques originally implemented for C~\cite{DeMarco2014, kim2013}. This is an oversight because Java and C are very different languages.

One way of addressing the limitations discussed above is to make automatic bug repair tools mimic the bug-fixing behavior of human programmers. This can be done by building a probabilistic model of the actions taken by programmers when they fix bugs.

%Our research aims to provide guidelines to improve one of these possible changes. The one which remains to be unstudied since it is probably the hardest of them: Replace. It will provide the data necessary for this tool and other approaches to have a guide on what is the way in which human programmers change their code when coding a fix for a bug. This way it will make it more likely for automatic error repair approaches to succeed in finding a patch for a particular error, and also, to provide automatic error repair software with heuristics to make the patches more human-like and therefore more readable and maintainable by human developers.

In this paper, we address the challenge of generating effective patches for bugs in Java programs whose fixes require statement replacements. We build a probabilistic model based on the statement replacement actions of human programmers in fixing bugs in Java programs. When a statement has to be replaced with another to fix a bug, the probabilistic model specifies the most likely candidates for the replacement operation. To build the probabilistic model, we conducted an an empirical study of 100 open source Java projects in GitHub and determined the frequency with which human programmers replace different statements in their source code in order to fix a bug. Our model can directly guide future research in automatic repair of Java programs, to increase the success rate of such techniques and the degree to which the patches are human-like and therefore more readable and maintainable by human developers. This is important, because patch quality is an important
concern in this area~\cite{Qi15}.

% On the other hand, some approaches~\cite{kim2013} search for specific patterns
% and modify source code according to predefined templates. A well known approach
% is PAR~\cite{kim2013}, which creates 10 different repair templates and applies
% them to the buggy code in an effort to repair it. In this paper we have taken 8
% out of the 10 PAR templates and tested how common they are in the repairs made
% by programmers in the latest official data dump of Github as of September 2015
% provided by~\cite{dyer2013}.  Our results provide evidence of how common those
% patterns are in practice. We take as reference the study performed by Dongsun
% Kim et al.~\cite{kim2013} in which they look for the most common ways in which
% programmers patch bugs in software. The researchers developed a variation of the
% tool Genprog~\cite{weimer2009,legoues2012} with several different templates
% resembling patterns programmers use to patch bugs, and then performed an
% empirical study to evaluate which patches do human programmers prefer.

\vspace{1ex} \noindent\textbf{Related work.} Zhong and Su~\cite{zhong2015} have studied the nature of bug fixes with a view toward improving existing program repair techniques and to understand their limitations. However, their study was limited to bug fixes from six Java projects. Our study is limited to replacements, but we study a significantly larger number of projects and we look at more statement types. Similar to our study, researchers have studied AST-level~\cite{Martinez:2015ez} and line-level changes~\cite{Asaduzzaman:2013df} across bug fixing commits. Although the granularity differs, our study is also novel with respect to the number of projects studied. Barr et al.~\cite{Barr2014} studied changes to Java programs to understand the ``Plastic Surgery Hypothesis" underlying certain types of program repair, an orthogonal concern. Kim et al.~\cite{kim2013} manually analyze changes to Java programs to inform automated repair, and show that doing so results in higher-quality repairs. Their results motivate studies of human repairs, as it may result in better patches. Long and Rinard learn probabilistic models from bug fixes to C code~\cite{Long2016}. Their model incorporates all possible operations that may be performed in order to fix a bug. Our model, on the other hand, describes only the replace operations that programmers may perform for fixing a bug. Moreover, it does not inform a new technique. Rather, it serves as a starting point for research on the repair of Java bugs. Soto et. al.~\cite{soto} use a heuristic to count statement replacements because of limitations of the Boa infrastructure~\cite{Dyer2013} in determining differences between file versions. We are able to count statement replacements in a more direct manner.

%Zhong and Su~\cite{zhong2015} ask
%some of the same questions we do, on 6 projects. We study a mutation operation that these researchers left out due to its complexity, which is the operator Replace, and we also look at more statement types than the previous work.. Similar to our study in
%Section~\ref{sec:stmtstudy}, researchers have 
%studied AST-level~\cite{Martinez:2015ez} and line-level
%changes~\cite{Asaduzzaman:2013df} across bug fixing commits.
%Although it is worth notice that so far none of these approaches have studied in depth the Replace operator.
%An important study was done in this subject by Soto et al~\cite{soto} in which they count the number of statements in the before-fix and after-fix versions of the code and try to infer a correlation between the two number of statements. This study avoids this assumption, and instead creates a direct comparison between the two abstract syntax trees to be able to infer more accurately when replacements are being made and create the model with this data.
