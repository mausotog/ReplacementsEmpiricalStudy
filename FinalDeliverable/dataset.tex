% !TEX root = FinalDeliverable.tex

\section{Dataset and Characteristics}
For the dataset of this study, we selected the top 100 Java projects in GitHub ranked according to the number of stars. In GitHub, when a user stars a project, it means that the user is:
\begin{itemize}
  \item Creating a bookmark for easier access
  \item Showing appreciation to the repository maintainer for their work
\end{itemize}

Therefore, by selecting the projects with higher number of stars, we are including those Java projects which are more liked and supported by the community of developers in GitHub. This includes projects like the Spring framework with which a lot of good developers are associated. As a result, building the probabilistic model based on a study of these projects would provide the benefit of making the model more reliable.
%, which may have a correlation with having a lot of followers, contributors and activity; and this diversity benefits our study since having a larger diversity of bug fixes, makes the model being built from this data more reliable and more diverse.

We restricted ourselves to studying only 20 recent commits because of time and storage limitations. Analyzing just 20 commits per project took a total of four days. Moreover, storing the before-fix and after-fix versions of each file in every commit for one project with about a 1000 commits required approximately 65GB of disk space. For these reasons, we have deferred a study of the full commit history of each of the projects as future work.

Determining whether a commit is a bug-fixing commit or not is usually done by applying a regular expression to the commit message. This method relies on the presence of certain words (that indicate that this commit might be related to a bug fix) and their order in the the commit message. We have used a regular expression that is similar to the one used by Boa~\cite{Dyer2013} in order to find bug-fixing commits. The regular expression that we have used is as follows:
\begin{itemize}
  \item "[Ff]ix(ed|es|ing)?(\textbackslash s)*([Bb]ug|[Ii]ssue|[Pp]roblem)?(s)?"
\end{itemize}

In order to ensure that the commits whose messages matched the above regular expression are actually bug-fixing commits, we performed the study in the following section.

% !TEX root = FinalDeliverable.tex

\subsection{Commit Validation}

In order to build the probabilistic model for statement replacements, it is necessary to ensure that
the commits selected using the regular expression specified above were
truly bug-fixing commits. In particular, some of the commits whose commit
messages matched the regular expression included changes to more
than five Java source code files. But prior studies of bug fixes~\cite{zhong2015}
have shown that programmers modify less than five files in order to fix
a bug. So, it became necessary to check if these commits which involved
modifications to five or more source code files were bug-fixing commits indeed.

We examined if commits that modified five or more Java files were bug-fixing
commits or not by manually inspecting such commits. We selected the 50 most recent
commits from the commit history of each of the 100 projects in our dataset. Some projects had less than
50 commits in total. We included all of the commits that were made to these projects.
Among these commits, we then proceeded to select the ones that involved modification
to five or more Java source files. We then manually inspected the commit messages
of the selected commits to determine if the changes were being made in order to
fix a bug. We included the commits which indicated that they were fixing issues
such as a crash, NullPointerException or some other issue identified using a project-specific ID.
We excluded the other commits. For example, if the commit message contains the words
``prefix" or ``suffix", it would match the regular expression that we have used, even though
it is not fixing a bug. We excluded all such commits.

The 100 projects that we have used in our study have a total of 62,470 commits.
We selected 1,353 of them for performing commit validation. These commits are
the ones among the 50 most recent commits for each of the 100 projects in which five or more source code files have been changed. We discovered that 847 out of these 1,353 commits
were actually bug-fixing commits. The remaining 506 commits were made for
other reasons. Majority of the commits that we have included for building the
probabilistic model are indeed bug-fixing commits. Table \ref{CommitValidationTable}
shows the distribution of bug-fixing commits in the 50 most recent commits for
the top 10 projects in terms of the total number of commits.

\begin{table}
\centering
\begin{tabular}{ |l|c|c| }
\hline
\thead{Project} & \thead{Bug-fixing \\ commits}    & \thead{Other \\ commits} \\
\hline
platform\_frameworks\_base	& 18	& 1	\\
\hline
kotlin		& 12	& 2	\\
\hline
elasticsearch	& 23	& 6	\\
\hline
netty	& 9	& 4	\\
\hline
libgdx & 8	& 3	\\
\hline
spring-boot	& 15	& 9	\\
\hline
spring-framework	& 12	& 22	\\
\hline
presto	& 17	& 7	\\
\hline
RxJava	& 22	& 10	\\
\hline
titan	& 12	& 5	\\
\hline
\end{tabular}
\caption{Distribution of bug-fixing commits in the 50 most recent commits for the top 10 projects with the highest number of commits}
\label{CommitValidationTable}

\end{table}

