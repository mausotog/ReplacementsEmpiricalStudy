% !TEX root = FinalDeliverable.tex

\section{Statement Replacements}
\label{sec:statement-replacements}

We build our probabilistic model at the granularity level of statements.
So, for each pair of statement kinds, we count the number of times a statement of the first kind
has been replaced with a statement of the second kind in the bug-fixing commits. In order to do this, we have used a tool named QACrashFix~\cite{Gao2015}. QACrashFix generates patches to bugs, particularly ones that result in crashes due to exceptions, by analyzing Q\&A sites or programmer forums. It makes use of the fact that programmers working on projects using the same framework (e.g., Android, Hadoop, Spring, etc.) make similar kinds of mistakes related to the use of the framework. If a bug causes an exception, more often than not, programmers get the help of experts in fixing it by posting to Q\&A sites. These posts usually include the exception message and stack trace as well. When another programmer encounters a similar exception, they can use QACrashFix which searches for posts made by other programmers on Q\&A sites and constructs patches from the solutions posted on the page.

The construction of patches by QACrashFix involves four steps. First, a search query is constructed based on information about the exception. This search query is used to obtain a list of Q\&A pages that contain possible fixes by performing a web search. The second step consists of extracting buggy and fixed code snippets from the Q\&A pages in the search results. For each pair of buggy and fixed code snippets, QACrashFix generates an edit script that describes how the Abstract Syntax Tree (AST) of the buggy code may be transformed to that of the fixed code. In the third step, potential patches are constructed by customizing the edit script to the code the programmer is searching a fix for. Finally, patches which fail to compile are eliminated.

We count statement replacements by generating an edit script that transforms the version of a source code file prior to a commit to the version of the same file after the commit. We use the edit script generation component of QACrashFix for this purpose. This component is based on another tool named GumTree~\cite{Falleri2014}. The edit scripts generated by GumTree contain four types of operations: (1) add, (2) delete, (3) update, and (4) move. QACrashFix extends GumTree to add two more operations: (1) replace, and (2) copy. The replace operation contains information about AST nodes which must be replaced with other AST nodes. We only consider replace operations in which both the AST node being replaced and the replacement AST node correspond to Java statements.