% This is "sig-alternate.tex" V2.1 April 2013
% This file should be compiled with V2.5 of "sig-alternate.cls" May 2012
%
% This example file demonstrates the use of the 'sig-alternate.cls'
% V2.5 LaTeX2e document class file. It is for those submitting
% articles to ACM Conference Proceedings WHO DO NOT WISH TO
% STRICTLY ADHERE TO THE SIGS (PUBS-BOARD-ENDORSED) STYLE.
% The 'sig-alternate.cls' file will produce a similar-looking,
% albeit, 'tighter' paper resulting in, invariably, fewer pages.
%
% ----------------------------------------------------------------------------------------------------------------
% This .tex file (and associated .cls V2.5) produces:
%       1) The Permission Statement
%       2) The Conference (location) Info information
%       3) The Copyright Line with ACM data
%       4) NO page numbers
%
% as against the acm_proc_article-sp.cls file which
% DOES NOT produce 1) thru' 3) above.
%
% Using 'sig-alternate.cls' you have control, however, from within
% the source .tex file, over both the CopyrightYear
% (defaulted to 200X) and the ACM Copyright Data
% (defaulted to X-XXXXX-XX-X/XX/XX).
% e.g.
% \CopyrightYear{2007} will cause 2007 to appear in the copyright line.
% \crdata{0-12345-67-8/90/12} will cause 0-12345-67-8/90/12 to appear in the copyright line.
%
% ---------------------------------------------------------------------------------------------------------------
% This .tex source is an example which *does* use
% the .bib file (from which the .bbl file % is produced).
% REMEMBER HOWEVER: After having produced the .bbl file,
% and prior to final submission, you *NEED* to 'insert'
% your .bbl file into your source .tex file so as to provide
% ONE 'self-contained' source file.
%
% ================= IF YOU HAVE QUESTIONS =======================
% Questions regarding the SIGS styles, SIGS policies and
% procedures, Conferences etc. should be sent to
% Adrienne Griscti (griscti@acm.org)
%
% Technical questions _only_ to
% Gerald Murray (murray@hq.acm.org)
% ===============================================================
%
% For tracking purposes - this is V2.0 - May 2012

\documentclass{sig-alternate-05-2015}
\usepackage{graphicx}
\usepackage[usenames,dvipsnames]{color}
\usepackage{soul} 
\usepackage{booktabs}

\newcommand{\todo}[1]
  {{\scriptsize \textbf{\color{red} {#1}}}}


\graphicspath{ {} }


\begin{document}

% Copyright
% \setcopyright{acmcopyright}
%\setcopyright{acmlicensed}
%\setcopyright{rightsretained}
%\setcopyright{usgov}
%\setcopyright{usgovmixed}
%\setcopyright{cagov}
%\setcopyright{cagovmixed}


\CopyrightYear{2016}
\setcopyright{acmcopyright}
\conferenceinfo{MSR'16,}{May 14-15, 2016, Austin, TX, USA}
\isbn{978-1-4503-4186-8/16/05}
\acmPrice{\$15.00}
\doi{http://dx.doi.org/10.1145/2901739.2903495}



% DOI
% \doi{10.475/123_4}

% ISBN
% \isbn{978-1-4503-4186-8}

%Conference
%\conferenceinfo{PLDI '13}{June 16--19, 2013, Seattle, WA, USA}

%\acmPrice{\$0.00}

%
% --- Author Metadata here ---
\conferenceinfo{MSR}{2016 Austin, Texas USA}
%\CopyrightYear{2007} % Allows default copyright year (20XX) to be over-ridden - IF NEED BE.
%\crdata{0-12345-67-8/90/01}  % Allows default copyright data (0-89791-88-6/97/05) to be over-ridden - IF NEED BE.
% --- End of Author Metadata ---

\title{Building a probabilistic model of statement replacements based on an empirical study of real Java bug fixes}

%
% You need the command \numberofauthors to handle the 'placement
% and alignment' of the authors beneath the title.
%
% For aesthetic reasons, we recommend 'three authors at a time'
% i.e. three 'name/affiliation blocks' be placed beneath the title.
%
% NOTE: You are NOT restricted in how many 'rows' of
% "name/affiliations" may appear. We just ask that you restrict
% the number of 'columns' to three.
%
% Because of the available 'opening page real-estate'
% we ask you to refrain from putting more than six authors
% (two rows with three columns) beneath the article title.
% More than six makes the first-page appear very cluttered indeed.
%
% Use the \alignauthor commands to handle the names
% and affiliations for an 'aesthetic maximum' of six authors.
% Add names, affiliations, addresses for
% the seventh etc. author(s) as the argument for the
% \additionalauthors command.
% These 'additional authors' will be output/set for you
% without further effort on your part as the last section in
% the body of your article BEFORE References or any Appendices.

\numberofauthors{1} %  in this sample file, there are a *total*
% of EIGHT authors. SIX appear on the 'first-page' (for formatting
% reasons) and the remaining two appear in the \additionalauthors section.
%
\author{
% You can go ahead and credit any number of authors here,
% e.g. one 'row of three' or two rows (consisting of one row of three
% and a second row of one, two or three).
%
% The command \alignauthor (no curly braces needed) should
% precede each author name, affiliation/snail-mail address and
% e-mail address. Additionally, tag each line of
% affiliation/address with \affaddr, and tag the
% e-mail address with \email.
%
% 1st. author
\alignauthor
Mauricio Soto, Selva Samuel\\
       \affaddr{School of Computer Science, Carnegie Mellon University, Pittsburgh, PA}\\
       \email{mauriciosoto@cmu.edu, ssamuel@andrew.cmu.edu} 
}
% There's nothing stopping you putting the seventh, eighth, etc.
% author on the opening page (as the 'third row') but we ask,
% for aesthetic reasons that you place these 'additional authors'
% in the \additional authors block, viz.
%\additionalauthors{Additional authors: John Smith (The Th{\o}rv{\"a}ld Group,
%email: {\texttt{jsmith@affiliation.org}}) and Julius P.~Kumquat
%(The Kumquat Consortium, email: {\texttt{jpkumquat@consortium.net}}).}
\date{February 2, 2016}
% Just remember to make sure that the TOTAL number of authors
% is the number that will appear on the first page PLUS the
% number that will appear in the \additionalauthors section.

\maketitle
\begin{abstract}
  Here goes the abstract
\end{abstract}


%
% The code below should be generated by the tool at
% http://dl.acm.org/ccs.cfm
% Please copy and paste the code instead of the example below. 
%
\begin{CCSXML}
<ccs2012>
 <concept>
  <concept_id>10010520.10010553.10010562</concept_id>
  <concept_desc>Computer systems organization~Embedded systems</concept_desc>
  <concept_significance>500</concept_significance>
 </concept>
 <concept>
  <concept_id>10010520.10010575.10010755</concept_id>
  <concept_desc>Computer systems organization~Redundancy</concept_desc>
  <concept_significance>300</concept_significance>
 </concept>
 <concept>
  <concept_id>10010520.10010553.10010554</concept_id>
  <concept_desc>Computer systems organization~Robotics</concept_desc>
  <concept_significance>100</concept_significance>
 </concept>
 <concept>
  <concept_id>10003033.10003083.10003095</concept_id>
  <concept_desc>Networks~Network reliability</concept_desc>
  <concept_significance>100</concept_significance>
 </concept>
</ccs2012>  
\end{CCSXML}


%
% End generated code
%

%
%  Use this command to print the description
%
\printccsdesc

% We no longer use \terms command
%\terms{Theory}

\keywords{Automatic error repair; Maintainability; Replacements}

\section{Introduction}


\section{Dataset and Characteristics}
How we have mined Github, the regex, etc


\subsection{Replacements and QACrashFix}


\section{Building the model}
\begin{table*}
	\centering
	\resizebox{\textwidth}{!}{
	%\renewcommand{\arraystretch}{0.9}% Tighter
		\begin{tabular}{l|rrrrrrrrrrrrrrr} \toprule
			&\texttt{Assert}&\texttt{Break}&\texttt{Continue}&\texttt{Do}&\texttt{For}&\texttt{If}&\texttt{Label}&\texttt{Return}&\texttt{Case}&\texttt{Switch}& \texttt{Synch}&\texttt{Throw}&\texttt{Try}&\texttt{TypeDecl}&\texttt{While}
			\\ \midrule
			
\texttt{Assert}	&-&7.48&3.76&0.53&8.30&23.05&0.31&20.04&4.90&4.62&1.30&13.50&7.23&0.03&4.95
\\
\texttt{Break} &1.00&-&4.08&0.60&9.93&26.03&0.13&25.39&2.48&1.57&1.79&8.39&11.73&0.10&6.77
\\
\texttt{Continue} &1.74&9.42&-&1.28&11.39&18.25&0.35&22.60&3.80&2.85&2.17&8.98&9.42&0.11&7.63
\\
\texttt{Do} &0.81&5.26&6.60&-&9.44&14.21&0.18&15.86&3.73&1.67&1.97&5.88&6.39&0.03&27.98
\\
\texttt{For} &0.86&6.28&3.19&0.79&-&22.89&0.09&21.08&5.01&3.34&1.87&10.01&10.71&0.08&13.79
\\
\texttt{If}&1.64&8.43&2.87&0.60&13.49&-&0.24&26.46&7.45&4.80&2.85&9.89&15.11&0.08&6.11
\\
\texttt{Label} &1.30&8.33&7.86&1.11&5.18&22.85&-&15.17&3.05&2.04&14.62&10.45&4.16&0.09&3.79
\\
\texttt{Return} &1.13&9.41&3.11&0.49&13.33&27.24&0.24&-&5.59&3.65&2.55&14.91&12.61&0.12&5.61
\\
\texttt{Case} &0.78&2.84&2.84&0.39&10.27&31.79&0.16&22.40&-&0.46&2.07&7.37&11.69&0.08&6.87
\\
\texttt{Switch}&1.14&2.72&3.80&0.55&11.07&34.14&0.13&21.86&0.75&-&1.53&8.65&9.02&0.05&4.58
\\
\texttt{Synch} &0.80&6.57&2.28&0.43&10.21&24.18&0.05&19.77&6.35&2.07&-&9.16&12.16&0.04&5.93
\\
\texttt{Throw} &2.11&6.57&2.58&0.48&11.87&18.84&0.17&32.28&4.64&3.30&2.74&-&10.08&0.07&4.27
\\
\texttt{Try}&0.71&7.41&3.02&0.66&11.73&27.75&0.11&23.24&5.63&2.65&2.58&8.99&-&0.09&5.42
\\
\texttt{TypeDecl} &0.00&4.51&7.52&1.00&10.28&21.05&0.50&17.79&6.02&1.75&2.01&9.27&11.53&-&6.77
\\
\texttt{While}& 0.72&8.02&3.82&1.96&23.16&19.78&0.12&16.48&6.56&3.09&1.64&6.81&7.80&0.04&-
			\\ \bottomrule
		\end{tabular}
		}
		\caption{Likelihood of replacing a statement type (row) by a statement of
          another type (column), for Java.}\label{tab:likeliness}
\end{table*}




\section{Evaluation}\label{sec:stmtstudy}




\section{Discussion and Future work}


\section{Acknowledgments}


\bibliographystyle{abbrv}
\bibliography{sigproc}  % sigproc.bib is the name of the Bibliography in this case
% You must have a proper ".bib" file
%  and remember to run:
% latex bibtex latex latex
% to resolve all references
%
% ACM needs 'a single self-contained file'!
%
%APPENDICES are optional
%\balancecolumns


%\balancecolumns % GM June 2007
% That's all folks!
\end{document}
