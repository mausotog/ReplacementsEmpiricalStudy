% !TEX root = FinalDeliverable.tex

\section{Conclusion and Future work}

In this work, we have begun to address two limitations of current automated program repair tools that are based on the generate-and-validate approach: (1) These tools generate a lot of patches that fail to compile. (2) Most of these tools are targeted to fixing bugs in C programs. In particular, our work aims to improve automated bug repair approaches for Java. For this purpose, we have built a probabilistic model that mimics the statement replacement actions of human programmers when they fix bugs in Java programs. We have evaluated our model using a 10-fold cross-validation approach. Our work is publicly available for download\footnote{https://github.com/mausotog/ReplacementsEmpiricalStudy}.

There are a number of ways in which this work can be extended. First, a larger number of projects and a larger number of commits for each project needs to be considered for building the probabilistic model. As stated earlier, we restricted the number of projects and commits in this study because of time and storage limitations. Analyzing a larger number of projects and commits would yield a better model.

Second, the effect of using a higher threshold for the number of prediction statements on the prediction performance of the model needs to be evaluated further. As mentioned in section~\ref{sec:stmtstudy}, we have used the 3 most likely statements according to our model when we need to predict the type of the statement that should be used to replace a given statement in order to fix a  bug. We suspect that using more number of statements would lead to better prediction.

Third, it would be interesting to incorporate this model into various generate-and-validate program repair tools, particularly GenProg in order to study the effectiveness of the model in generating patches.
